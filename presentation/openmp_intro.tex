\documentclass[handout]{beamer}
\usecolortheme{beaver}
\setbeamertemplate{navigation symbols}{}

\usepackage{tikz}

\usepackage[outputdir=build]{minted}
\usepackage{tcolorbox}
\tcbuselibrary{minted,skins}

% Patch to fix https://github.com/T-F-S/tcolorbox/issues/12
\makeatletter
\def\tcb@minted@input@listing#1#2#3#4{%
  \edef\temp@a{#4}%
  \ifx\temp@a\@empty%
  \else%
    \toks@=\expandafter{#4}%
    \edef\tcb@temp{\noexpand\usemintedstyle{\the\toks@}}%
    \tcb@temp%
  \fi%
  \toks@=\expandafter{#1}%
  \edef\tcb@temp{\noexpand\inputminted[\the\toks@]}%
  \IfFileExists{\minted@outputdir#3}%
    {\tcb@temp{#2}{\minted@outputdir#3}}%
    {\tcb@temp{#2}{#3}}%
}
\makeatother

\newtcblisting{ccode}[2][]{%
  listing engine=minted,
  minted language=#2,
  minted options={breaklines,breakanywhere,fontsize=\scriptsize,gobble=8},
  listing only,
  before skip=0pt,
  after skip=8pt,
  left skip=0pt,
  right skip=0pt,
  size=fbox,
  sharp corners,
  %colframe=white!75!black,
  colframe=white,
  boxrule=0pt,
  frame hidden,
  #1
}


\newtcbinputlisting[]{\inputcode}[3][]{%
  listing engine=minted,
  minted language=#2,
  minted options={breaklines,breakanywhere,fontsize=\scriptsize,#1},
  listing file={#3},
  listing only,
  size=fbox,
  before skip=0pt,
  after skip=10pt,
  left skip=0pt,
  right skip=0pt,
  sharp corners,
  %colframe=white!75!black,
  colframe=white,
  boxrule=0pt,
  frame hidden
}


\title{Introduction to OpenMP}
\subtitle{Ferienakademie 2017}
\author{Nathan Brei}
\institute{Technical University of Munich}
\date\today

\begin{document}
\begin{frame}
  \titlepage
\end{frame}

\begin{frame}
  \frametitle{Register Machine Model}
  \begin{center}
    \begin{tikzpicture}[scale=0.5]
      \draw[thick,fill=lightgray] (0,0) rectangle (8,10);

      \draw[thick] (2,1) -- (2,9);
      \draw[thick] (2,1) -- (8,1);
      \draw[thick] (2,2) -- (8,2);
      \draw[thick] (2,3) -- (8,3);
      \draw[thick] (2,4) -- (8,4);
      \draw[thick] (2,5) -- (8,5);
      \draw[thick] (2,6) -- (8,6);
      \draw[thick] (2,7) -- (8,7);
      \draw[thick] (2,8) -- (8,8);
      \draw[thick] (2,9) -- (8,9);

      \node[above left] at (2,8) {0};
      \node[above left] at (2,7) {4};
      \node[above left] at (2,6) {8};
      \node[above left] at (2,5) {12};
      \node[above left] at (2,4) {16};
      \node[above left] at (2,3) {20};
      \node[above left] at (2,2) {24};
      \node[above left] at (2,1) {\vdots};
      %\draw[help lines] (0,0) grid (10,10);
    \end{tikzpicture}
  \end{center}
\end{frame}


\begin{frame}[fragile]
  \frametitle{Launching a team of threads --- Parallel Regions}
  \begin{columns}[t]%[onlytextwidth]
    \begin{column}{0.65\textwidth}
      \inputcode[]{c}{src/ex1.c}
    \end{column}
    \pause
    \begin{column}{0.4\textwidth}
      \inputcode[]{text}{output/ex1.txt}
      \begin{ccode}[]
        {text}
        About to fork...
        Hello from thread 2 of 3
        Hello from thread 0 of 3
        Hello from thread 1 of 3
        ...Rejoined.
      \end{ccode}
    \end{column}
  \end{columns}
  \pause
  \begin{itemize}
  \item \emph{Fork-join execution model}
  \item Print statements are interleaved nondeterministically
  \item Number of threads is chosen automatically, or specified by \mintinline{text}{$OMP_NUM_THREADS}, \mintinline{c}{omp_set_num_threads(int t)}, \mintinline{c}{#pragma omp parallel num_threads(2)}
  \end{itemize}
\end{frame}

\begin{frame}[fragile]
  \frametitle{Assigning code blocks to threads --- Parallel Sections}
  \begin{columns}[t]%[onlytextwidth]
    \begin{column}{0.65\textwidth}
      \inputcode[]{c}{src/ex2.c}
    \end{column}
    \pause
    \begin{column}{0.4\textwidth}
      \begin{ccode}[]
        {text}
        Thread 0, iter 0
        Thread 0, iter 1
          Thread 1, iter 0
        Thread 0, iter 2
          Thread 1, iter 1
          Thread 1, iter 2
          Thread 1, iter 3
        Thread 0, iter 3\end{ccode}
      \inputcode[]{text}{output/ex2.txt}
    \end{column}
  \end{columns}
  \pause
  \begin{itemize}
  \item Number of parallel sections is fixed at compile time.
  \item No guarantee how statements get interleaved, or how many threads are actually used!
  \end{itemize}
\end{frame}

\begin{frame}[fragile]
  \frametitle{Problem: Race conditions}
  Definition: A class of bug where the program's output depends on the timing of events which are not under the programmer's control.  
  \begin{columns}[t]%[onlytextwidth]
    \begin{column}{0.5\textwidth}
      \begin{ccode}[]{c}
        int a = 0;
        int b = 0;
        #pragma omp parallel sections
        {
          #pragma omp section
          a = 1;

          #pragma omp section
          b = a;
        }
        printf("b = %d\n", b);\end{ccode}
    \end{column}
    \pause
    \begin{column}{0.3\textwidth}
      \begin{ccode}[]
        {text}
        a = 0
        b = 0
        a = 1
        b = a
          ==> (b == 1)\end{ccode}
      \begin{ccode}[]
        {text}
        a = 0
        b = 0
        b = a
        a = 1
          ==> (b == 0)\end{ccode}
    \end{column}
  \end{columns}
  \pause
  \begin{itemize}
  \item For simple programs, the output often \emph{appears} deterministic.
  \item However, instruction interleaving is \emph{not} defined by OpenMP.
  \item It depends on the CPU architecture, operating system, system load, compiler, and code optimization level.
  \item Race conditions often lie dormant only to materialize when you least want them to.
  \end{itemize}

\end{frame}



\begin{frame}[fragile]
  \frametitle{Problem: Race conditions}
  Some race conditions are not obvious from the C code, but emerge when compiled to assembly/microcode. \emph{Every} variable access in C requires memory \texttt{load} or \texttt{store} operations unless the compiler can prove otherwise. 
  
  \begin{columns}[t]%[onlytextwidth]
    \begin{column}{0.43\textwidth}
      \inputcode[firstline=5,lastline=22,gobble=2]{c}{src/ex4.c}
    \end{column}

    \begin{column}{0.43\textwidth}
      \inputcode[firstline=5,lastline=22,gobble=2]{c}{src/ex5.c}
    \end{column}
    
    \begin{column}{0.3\textwidth}
      \begin{ccode}[after skip=4pt]
        {text}
        load (total) -> a
          load (total) -> c
        add 1, a -> b
          add 2, c -> d
        store b -> (total)
          store d -> (total)

          ==> (total == 2)\end{ccode}
      \begin{ccode}[]
        {text}
        load (total) -> a
        add 1, a -> b
          load (total) -> c
          add 2, c -> d
          store d -> (total)
        store b -> (total)
        
          ==> (total == 1)\end{ccode}
    \end{column}
  \end{columns}
\end{frame}



\begin{frame}[fragile]
  \frametitle{Privatizing intermediate variables}
  The first technique for eliminating race conditions is distinguishing between the variables which \emph{must} be shared and those which can be made local to a thread. These variables should be declared private.
  \begin{columns}[t]%[onlytextwidth]
    \begin{column}{0.4\textwidth}
      \inputcode[firstline=5,lastline=22,gobble=2]{c}{src/ex4.c}
    \end{column}
    \pause
    \begin{column}{0.4\textwidth}
      \inputcode[]{text}{output/ex1.txt}
      \begin{ccode}[]
        {c}
        Another example
        Yet another example
        PENIS
        PENIS
      \end{ccode}
    \end{column}
  \end{columns}
  \pause
  \begin{itemize}
  \item 
  \end{itemize}
\end{frame}


\begin{frame}[fragile]
  \frametitle{Protecting access to shared variables --- Critical and Atomic Sections}
  \begin{columns}[t]%[onlytextwidth]
    \begin{column}{0.65\textwidth}
      \inputcode[]{c}{src/ex1.c}
    \end{column}
    \pause
    \begin{column}{0.4\textwidth}
      \inputcode[]{text}{output/ex1.txt}
      \begin{ccode}[]
        {c}
        Another example
        Yet another example
        PENIS
        PENIS
      \end{ccode}
    \end{column}
  \end{columns}
  \pause
  \begin{itemize}
  \item Mark a region as access only by mutual exclusion
  \item Underlying idea: threads use a mutex variable to indicate that
  \item Amdahl's law: (Over) using critical regions hurts scalability 
  \end{itemize}
\end{frame}



\begin{frame}[fragile]
  \frametitle{Enforcing an ordering between threads --- Barriers}
  \begin{columns}[t]%[onlytextwidth]
    \begin{column}{0.65\textwidth}
      \inputcode[]{c}{src/ex1.c}
    \end{column}
    \pause
    \begin{column}{0.4\textwidth}
      \inputcode[]{text}{output/ex1.txt}
      \begin{ccode}[]
        {c}
        Another example
        Yet another example
        PENIS
        PENIS
      \end{ccode}
    \end{column}
  \end{columns}
  \pause
  \begin{itemize}
  \item Print statements are interleaved nondeterministically
  \end{itemize}
\end{frame}


\begin{frame}[fragile]
  \frametitle{Dependency Graphs}
  \begin{columns}[t]%[onlytextwidth]
    \begin{column}{0.65\textwidth}
      \inputcode[]{c}{src/ex1.c}
    \end{column}
    \pause
    \begin{column}{0.4\textwidth}
      \inputcode[]{text}{output/ex1.txt}
      \begin{ccode}[]
        {c}
        Another example
        Yet another example
        PENIS
        PENIS
      \end{ccode}
    \end{column}
  \end{columns}
  \pause
  \begin{itemize}
  \item Print statements are interleaved nondeterministically
  \end{itemize}
\end{frame}



\begin{frame}[fragile]
  \frametitle{Parallelizing For Loops}
  For numerical code, most of the computation usually happens inside an inner loop, and we wish to distribute the work evenly across an arbitrary number of threads determined at runtime.
  \begin{columns}[t]%[onlytextwidth]
    \begin{column}{0.65\textwidth}
      \inputcode[]{c}{src/exfor.c}
    \end{column}
    \pause
    \begin{column}{0.4\textwidth}
      \inputcode[]{text}{output/exfor.txt}
    \end{column}
  \end{columns}
  \pause
  \begin{itemize}
  \item Available scheduling strategies: static, dynamic, guided
  \item Adjust chunk size in order to increase the parallelism granularity
  \item  
  \end{itemize}
\end{frame}



\begin{frame}[fragile]
  \frametitle{Loop-Carried Dependencies}
  \begin{columns}[t]%[onlytextwidth]
    \begin{column}{0.65\textwidth}
      \inputcode[]{c}{src/ex1.c}
    \end{column}
    \pause
    \begin{column}{0.4\textwidth}
      \inputcode[]{text}{output/ex1.txt}
      \begin{ccode}[]
        {c}
        Another example
        Yet another example
        PENIS

      \end{ccode}
    \end{column}
  \end{columns}
  \pause
  \begin{itemize}
  \item Heuristic: Array accesses such as \texttt{A[i]} are good, \texttt{A[i-1]} are bad
  \item Remove dependencies via loop splitting, alignment, interchange
  \end{itemize}
\end{frame}

\begin{frame}[fragile]
  \frametitle{Putting everything together: Sum all elements in array}
  \begin{columns}[t]%[onlytextwidth]
    \begin{column}{0.65\textwidth}
      \inputcode[]{c}{src/ex1.c}
    \end{column}
    \pause
    \begin{column}{0.4\textwidth}
      \inputcode[]{text}{output/ex1.txt}
      \begin{ccode}[]
        {c}
        Another example
        Yet another example
        PENIS
        PENIS
      \end{ccode}
    \end{column}
  \end{columns}
  \pause
  \begin{itemize}
  \item Heuristic: Array accesses such as \texttt{A[i]} are good, \texttt{A[i-1]} are bad
  \item Remove dependencies via loop splitting, alignment, interchange
  \end{itemize}
\end{frame}
\end{document}
